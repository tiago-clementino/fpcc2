\documentclass[]{article}
\usepackage{lmodern}
\usepackage{amssymb,amsmath}
\usepackage{ifxetex,ifluatex}
\usepackage{fixltx2e} % provides \textsubscript
\ifnum 0\ifxetex 1\fi\ifluatex 1\fi=0 % if pdftex
  \usepackage[T1]{fontenc}
  \usepackage[utf8]{inputenc}
\else % if luatex or xelatex
  \ifxetex
    \usepackage{mathspec}
  \else
    \usepackage{fontspec}
  \fi
  \defaultfontfeatures{Ligatures=TeX,Scale=MatchLowercase}
\fi
% use upquote if available, for straight quotes in verbatim environments
\IfFileExists{upquote.sty}{\usepackage{upquote}}{}
% use microtype if available
\IfFileExists{microtype.sty}{%
\usepackage{microtype}
\UseMicrotypeSet[protrusion]{basicmath} % disable protrusion for tt fonts
}{}
\usepackage[margin=1in]{geometry}
\usepackage{hyperref}
\hypersetup{unicode=true,
            pdftitle={LPCC02 - Lab 2 - Checkpoint 1},
            pdfborder={0 0 0},
            breaklinks=true}
\urlstyle{same}  % don't use monospace font for urls
\usepackage{color}
\usepackage{fancyvrb}
\newcommand{\VerbBar}{|}
\newcommand{\VERB}{\Verb[commandchars=\\\{\}]}
\DefineVerbatimEnvironment{Highlighting}{Verbatim}{commandchars=\\\{\}}
% Add ',fontsize=\small' for more characters per line
\usepackage{framed}
\definecolor{shadecolor}{RGB}{248,248,248}
\newenvironment{Shaded}{\begin{snugshade}}{\end{snugshade}}
\newcommand{\KeywordTok}[1]{\textcolor[rgb]{0.13,0.29,0.53}{\textbf{#1}}}
\newcommand{\DataTypeTok}[1]{\textcolor[rgb]{0.13,0.29,0.53}{#1}}
\newcommand{\DecValTok}[1]{\textcolor[rgb]{0.00,0.00,0.81}{#1}}
\newcommand{\BaseNTok}[1]{\textcolor[rgb]{0.00,0.00,0.81}{#1}}
\newcommand{\FloatTok}[1]{\textcolor[rgb]{0.00,0.00,0.81}{#1}}
\newcommand{\ConstantTok}[1]{\textcolor[rgb]{0.00,0.00,0.00}{#1}}
\newcommand{\CharTok}[1]{\textcolor[rgb]{0.31,0.60,0.02}{#1}}
\newcommand{\SpecialCharTok}[1]{\textcolor[rgb]{0.00,0.00,0.00}{#1}}
\newcommand{\StringTok}[1]{\textcolor[rgb]{0.31,0.60,0.02}{#1}}
\newcommand{\VerbatimStringTok}[1]{\textcolor[rgb]{0.31,0.60,0.02}{#1}}
\newcommand{\SpecialStringTok}[1]{\textcolor[rgb]{0.31,0.60,0.02}{#1}}
\newcommand{\ImportTok}[1]{#1}
\newcommand{\CommentTok}[1]{\textcolor[rgb]{0.56,0.35,0.01}{\textit{#1}}}
\newcommand{\DocumentationTok}[1]{\textcolor[rgb]{0.56,0.35,0.01}{\textbf{\textit{#1}}}}
\newcommand{\AnnotationTok}[1]{\textcolor[rgb]{0.56,0.35,0.01}{\textbf{\textit{#1}}}}
\newcommand{\CommentVarTok}[1]{\textcolor[rgb]{0.56,0.35,0.01}{\textbf{\textit{#1}}}}
\newcommand{\OtherTok}[1]{\textcolor[rgb]{0.56,0.35,0.01}{#1}}
\newcommand{\FunctionTok}[1]{\textcolor[rgb]{0.00,0.00,0.00}{#1}}
\newcommand{\VariableTok}[1]{\textcolor[rgb]{0.00,0.00,0.00}{#1}}
\newcommand{\ControlFlowTok}[1]{\textcolor[rgb]{0.13,0.29,0.53}{\textbf{#1}}}
\newcommand{\OperatorTok}[1]{\textcolor[rgb]{0.81,0.36,0.00}{\textbf{#1}}}
\newcommand{\BuiltInTok}[1]{#1}
\newcommand{\ExtensionTok}[1]{#1}
\newcommand{\PreprocessorTok}[1]{\textcolor[rgb]{0.56,0.35,0.01}{\textit{#1}}}
\newcommand{\AttributeTok}[1]{\textcolor[rgb]{0.77,0.63,0.00}{#1}}
\newcommand{\RegionMarkerTok}[1]{#1}
\newcommand{\InformationTok}[1]{\textcolor[rgb]{0.56,0.35,0.01}{\textbf{\textit{#1}}}}
\newcommand{\WarningTok}[1]{\textcolor[rgb]{0.56,0.35,0.01}{\textbf{\textit{#1}}}}
\newcommand{\AlertTok}[1]{\textcolor[rgb]{0.94,0.16,0.16}{#1}}
\newcommand{\ErrorTok}[1]{\textcolor[rgb]{0.64,0.00,0.00}{\textbf{#1}}}
\newcommand{\NormalTok}[1]{#1}
\usepackage{longtable,booktabs}
\usepackage{graphicx,grffile}
\makeatletter
\def\maxwidth{\ifdim\Gin@nat@width>\linewidth\linewidth\else\Gin@nat@width\fi}
\def\maxheight{\ifdim\Gin@nat@height>\textheight\textheight\else\Gin@nat@height\fi}
\makeatother
% Scale images if necessary, so that they will not overflow the page
% margins by default, and it is still possible to overwrite the defaults
% using explicit options in \includegraphics[width, height, ...]{}
\setkeys{Gin}{width=\maxwidth,height=\maxheight,keepaspectratio}
\IfFileExists{parskip.sty}{%
\usepackage{parskip}
}{% else
\setlength{\parindent}{0pt}
\setlength{\parskip}{6pt plus 2pt minus 1pt}
}
\setlength{\emergencystretch}{3em}  % prevent overfull lines
\providecommand{\tightlist}{%
  \setlength{\itemsep}{0pt}\setlength{\parskip}{0pt}}
\setcounter{secnumdepth}{0}
% Redefines (sub)paragraphs to behave more like sections
\ifx\paragraph\undefined\else
\let\oldparagraph\paragraph
\renewcommand{\paragraph}[1]{\oldparagraph{#1}\mbox{}}
\fi
\ifx\subparagraph\undefined\else
\let\oldsubparagraph\subparagraph
\renewcommand{\subparagraph}[1]{\oldsubparagraph{#1}\mbox{}}
\fi

%%% Use protect on footnotes to avoid problems with footnotes in titles
\let\rmarkdownfootnote\footnote%
\def\footnote{\protect\rmarkdownfootnote}

%%% Change title format to be more compact
\usepackage{titling}

% Create subtitle command for use in maketitle
\newcommand{\subtitle}[1]{
  \posttitle{
    \begin{center}\large#1\end{center}
    }
}

\setlength{\droptitle}{-2em}
  \title{LPCC02 - Lab 2 - Checkpoint 1}
  \pretitle{\vspace{\droptitle}\centering\huge}
  \posttitle{\par}
  \author{}
  \preauthor{}\postauthor{}
  \date{}
  \predate{}\postdate{}


\begin{document}
\maketitle

\textbf{Tiago Lucas Pereira Clementino}

\textbf{15 de abril de 2018}

\subsubsection{Bibliotecas utilizadas}\label{bibliotecas-utilizadas}

\begin{Shaded}
\begin{Highlighting}[]
\KeywordTok{library}\NormalTok{(tidyverse)}
\KeywordTok{library}\NormalTok{(here)}
\KeywordTok{library}\NormalTok{(knitr)}
\KeywordTok{library}\NormalTok{(gridExtra)}
\KeywordTok{library}\NormalTok{(ggExtra)}
\KeywordTok{library}\NormalTok{(gapminder)}
\KeywordTok{library}\NormalTok{(ggalt)}
\KeywordTok{theme_set}\NormalTok{(}\KeywordTok{theme_bw}\NormalTok{())}
\KeywordTok{options}\NormalTok{(}\DataTypeTok{warn=}\OperatorTok{-}\DecValTok{1}\NormalTok{)}
\end{Highlighting}
\end{Shaded}

\section{Contexto}\label{contexto}

Nosso objetivo é analisar e extrair informação de dados coletados por
uma plataforma aberta de integração contínua chamada Travistorrent. O
Travisttorrent é um serviço de integração contínua de projetos
disponíveis no Github, um repositório de projetos colaborativos também
aberto.

A integração contínua é um processo onde o desenvolvedor integra o
código alterado e/ou criado ao projeto principal na mesma frequência com
que as funcionalidades são introduzidas.

Nossa intenção é analisar os dados fazendo um paralelo entre as
linguagens de programação Java e Ruby, levantando questões inerentes às
características dos projetos de software que as utilizam. Diante disto,
descartamos quaisquer dados referentes a outras linguagens.

\section{Nossos Dados}\label{nossos-dados}

Os dados são informações diversas, referentes a projetos disponíveis no
Github e que, no momento da coleta, utilizaram o Travistorent nos
últimos três meses, além de corresponder a certas especificações de
filtragem. Estas informações descrevem operações inerentes ao andamento
de projetos no Github e a procedimentos de integração contínua (testes,
\emph{commits}, PR, \emph{builds} de integração, etc) ao longo de um
certo período de tempo.

\subsubsection{Carregando os dados}\label{carregando-os-dados}

\begin{Shaded}
\begin{Highlighting}[]
\NormalTok{projetos =}\StringTok{ }\KeywordTok{read_csv}\NormalTok{(here}\OperatorTok{::}\KeywordTok{here}\NormalTok{(}\StringTok{"dados/projetos.csv"}\NormalTok{))}
\end{Highlighting}
\end{Shaded}

\begin{verbatim}
## Parsed with column specification:
## cols(
##   gh_project_name = col_character(),
##   team = col_double(),
##   lang = col_character(),
##   sloc_end = col_integer(),
##   sloc_med = col_double(),
##   activity_period = col_integer(),
##   num_commits = col_integer(),
##   commits_per_month = col_double(),
##   tests_per_kloc = col_double(),
##   total_builds = col_integer(),
##   build_success_prop = col_double(),
##   builds_per_month = col_double(),
##   tests_added_per_build = col_double(),
##   tests_successful = col_double(),
##   test_density = col_double(),
##   test_size_avg = col_double()
## )
\end{verbatim}

\subsubsection{Filtrando os dados (eliminando linguagens diferentes de
Java e Ruby
identificadas)}\label{filtrando-os-dados-eliminando-linguagens-diferentes-de-java-e-ruby-identificadas}

\begin{Shaded}
\begin{Highlighting}[]
\NormalTok{projetos =}\StringTok{ }\NormalTok{projetos }\OperatorTok\StringTok{ }
\StringTok{    }\KeywordTok{filter}\NormalTok{(lang }\OperatorTok{!=}\StringTok{ "javascript"}\NormalTok{)}
\end{Highlighting}
\end{Shaded}

\section{Variáveis}\label{variaveis}

Nossa base de dados conta com variáveis bem intuitivas.

\begin{Shaded}
\begin{Highlighting}[]
\NormalTok{projetos }\OperatorTok\StringTok{ }\KeywordTok{names}\NormalTok{()}
\end{Highlighting}
\end{Shaded}

\begin{verbatim}
##  [1] "gh_project_name"       "team"                 
##  [3] "lang"                  "sloc_end"             
##  [5] "sloc_med"              "activity_period"      
##  [7] "num_commits"           "commits_per_month"    
##  [9] "tests_per_kloc"        "total_builds"         
## [11] "build_success_prop"    "builds_per_month"     
## [13] "tests_added_per_build" "tests_successful"     
## [15] "test_density"          "test_size_avg"
\end{verbatim}

Cada uma delas descreve alguma característica dos projetos.

\begin{longtable}[]{@{}lll@{}}
\toprule
\begin{minipage}[b]{0.21\columnwidth}\raggedright\strut
Nome\strut
\end{minipage} & \begin{minipage}[b]{0.61\columnwidth}\raggedright\strut
Descrição\strut
\end{minipage} & \begin{minipage}[b]{0.10\columnwidth}\raggedright\strut
Tipo\strut
\end{minipage}\tabularnewline
\midrule
\endhead
\begin{minipage}[t]{0.21\columnwidth}\raggedright\strut
\textbf{gh\_project\_name}\strut
\end{minipage} & \begin{minipage}[t]{0.61\columnwidth}\raggedright\strut
nome do projeto\strut
\end{minipage} & \begin{minipage}[t]{0.10\columnwidth}\raggedright\strut
Categórica\strut
\end{minipage}\tabularnewline
\begin{minipage}[t]{0.21\columnwidth}\raggedright\strut
\textbf{team}\strut
\end{minipage} & \begin{minipage}[t]{0.61\columnwidth}\raggedright\strut
total de desenvolvedores que participaram do projeto até sua última
medição\strut
\end{minipage} & \begin{minipage}[t]{0.10\columnwidth}\raggedright\strut
Numérica\strut
\end{minipage}\tabularnewline
\begin{minipage}[t]{0.21\columnwidth}\raggedright\strut
\textbf{lang}\strut
\end{minipage} & \begin{minipage}[t]{0.61\columnwidth}\raggedright\strut
linguagem de programação predominante\strut
\end{minipage} & \begin{minipage}[t]{0.10\columnwidth}\raggedright\strut
Categórica\strut
\end{minipage}\tabularnewline
\begin{minipage}[t]{0.21\columnwidth}\raggedright\strut
\textbf{sloc\_end}\strut
\end{minipage} & \begin{minipage}[t]{0.61\columnwidth}\raggedright\strut
total de linhas de código na última medição do projeto\strut
\end{minipage} & \begin{minipage}[t]{0.10\columnwidth}\raggedright\strut
Numérica\strut
\end{minipage}\tabularnewline
\begin{minipage}[t]{0.21\columnwidth}\raggedright\strut
\textbf{sloc\_med}\strut
\end{minipage} & \begin{minipage}[t]{0.61\columnwidth}\raggedright\strut
total de linhas de código no meio do tempo de atividade estimado do
projeto\strut
\end{minipage} & \begin{minipage}[t]{0.10\columnwidth}\raggedright\strut
Numérica\strut
\end{minipage}\tabularnewline
\begin{minipage}[t]{0.21\columnwidth}\raggedright\strut
\textbf{activity\_period}\strut
\end{minipage} & \begin{minipage}[t]{0.61\columnwidth}\raggedright\strut
tempo de atividade estimado do projeto\strut
\end{minipage} & \begin{minipage}[t]{0.10\columnwidth}\raggedright\strut
Numérica\strut
\end{minipage}\tabularnewline
\begin{minipage}[t]{0.21\columnwidth}\raggedright\strut
\textbf{num\_commits}\strut
\end{minipage} & \begin{minipage}[t]{0.61\columnwidth}\raggedright\strut
total de submissões de alteração durante todo o tempo de atividade do
projeto\strut
\end{minipage} & \begin{minipage}[t]{0.10\columnwidth}\raggedright\strut
Numérica\strut
\end{minipage}\tabularnewline
\begin{minipage}[t]{0.21\columnwidth}\raggedright\strut
\textbf{commits\_per\_month}\strut
\end{minipage} & \begin{minipage}[t]{0.61\columnwidth}\raggedright\strut
total de submissões por mês\strut
\end{minipage} & \begin{minipage}[t]{0.10\columnwidth}\raggedright\strut
Numérica\strut
\end{minipage}\tabularnewline
\begin{minipage}[t]{0.21\columnwidth}\raggedright\strut
\textbf{tests\_per\_kloc}\strut
\end{minipage} & \begin{minipage}[t]{0.61\columnwidth}\raggedright\strut
casos de teste por total de linhas de código\strut
\end{minipage} & \begin{minipage}[t]{0.10\columnwidth}\raggedright\strut
Numérica\strut
\end{minipage}\tabularnewline
\begin{minipage}[t]{0.21\columnwidth}\raggedright\strut
\textbf{total\_builds}\strut
\end{minipage} & \begin{minipage}[t]{0.61\columnwidth}\raggedright\strut
total de integrações\strut
\end{minipage} & \begin{minipage}[t]{0.10\columnwidth}\raggedright\strut
Numérica\strut
\end{minipage}\tabularnewline
\begin{minipage}[t]{0.21\columnwidth}\raggedright\strut
\textbf{build\_success\_prop}\strut
\end{minipage} & \begin{minipage}[t]{0.61\columnwidth}\raggedright\strut
proporção de integrações bem-sucedidas\strut
\end{minipage} & \begin{minipage}[t]{0.10\columnwidth}\raggedright\strut
Numérica\strut
\end{minipage}\tabularnewline
\begin{minipage}[t]{0.21\columnwidth}\raggedright\strut
\textbf{builds\_per\_month}\strut
\end{minipage} & \begin{minipage}[t]{0.61\columnwidth}\raggedright\strut
total médio de integrações por mês\strut
\end{minipage} & \begin{minipage}[t]{0.10\columnwidth}\raggedright\strut
Numérica\strut
\end{minipage}\tabularnewline
\begin{minipage}[t]{0.21\columnwidth}\raggedright\strut
\textbf{tests\_added\_per\_build}\strut
\end{minipage} & \begin{minipage}[t]{0.61\columnwidth}\raggedright\strut
total médio de testes adicionados por integração\strut
\end{minipage} & \begin{minipage}[t]{0.10\columnwidth}\raggedright\strut
Numérica\strut
\end{minipage}\tabularnewline
\begin{minipage}[t]{0.21\columnwidth}\raggedright\strut
\textbf{tests\_successful}\strut
\end{minipage} & \begin{minipage}[t]{0.61\columnwidth}\raggedright\strut
total de testes bem-sucedidos\strut
\end{minipage} & \begin{minipage}[t]{0.10\columnwidth}\raggedright\strut
Numérica\strut
\end{minipage}\tabularnewline
\begin{minipage}[t]{0.21\columnwidth}\raggedright\strut
\textbf{test\_density}\strut
\end{minipage} & \begin{minipage}[t]{0.61\columnwidth}\raggedright\strut
densidade de testes\strut
\end{minipage} & \begin{minipage}[t]{0.10\columnwidth}\raggedright\strut
Numérica\strut
\end{minipage}\tabularnewline
\begin{minipage}[t]{0.21\columnwidth}\raggedright\strut
\textbf{test\_size\_avg}\strut
\end{minipage} & \begin{minipage}[t]{0.61\columnwidth}\raggedright\strut
tamanho médio dos casos de testes\strut
\end{minipage} & \begin{minipage}[t]{0.10\columnwidth}\raggedright\strut
Numérica\strut
\end{minipage}\tabularnewline
\bottomrule
\end{longtable}

Na nossa análise buscaremos uma ou mais características da relação entre
as variáveis \textbf{sloc\_end} e \textbf{sloc\_med}, e
\textbf{tests\_per\_kloc} e \textbf{test\_size\_avg} explorando marcas e
canais gráficos. Além de incluirmos \textbf{activity\_period},
\textbf{team} e \textbf{lang} eventualmente para posicionar nossas
análises no nosso conjunto de dados.

\section{Distribuições dos dados entre Java e
Ruby}\label{distribuicoes-dos-dados-entre-java-e-ruby}

O gráfico abaixo apresenta a distribuição dos dados entre Java e Ruby. É
fácil perceber que a vantagem numérica de Ruby em relação à Java é
grande, mas isto não deve interferir em nossas observações. Nossas
análises farão um paralelo entre estas duas linguagens.

\begin{Shaded}
\begin{Highlighting}[]
\NormalTok{projetos }\OperatorTok\StringTok{ }
\StringTok{  }\KeywordTok{group_by}\NormalTok{(lang) }\OperatorTok\StringTok{ }
\StringTok{  }\KeywordTok{summarise}\NormalTok{(}\DataTypeTok{projetos_ =} \KeywordTok{n}\NormalTok{()) }\OperatorTok\StringTok{ }
\StringTok{  }\KeywordTok{ggplot}\NormalTok{(}\KeywordTok{aes}\NormalTok{(}\DataTypeTok{x =}\NormalTok{ lang, }\DataTypeTok{y =}\NormalTok{ projetos_)) }\OperatorTok{+}\StringTok{ }
\StringTok{  }\KeywordTok{geom_col}\NormalTok{(}\DataTypeTok{fill =} \StringTok{"darkcyan"}\NormalTok{, }\DataTypeTok{color =} \StringTok{"darkcyan"}\NormalTok{) }\OperatorTok{+}
\StringTok{  }\KeywordTok{labs}\NormalTok{(}\DataTypeTok{x=}\OtherTok{NULL}\NormalTok{,  }
    \DataTypeTok{y=}\StringTok{'Projetos'}\NormalTok{, }
    \DataTypeTok{title=}\StringTok{"Dados por Linguagem (Java x Ruby)"}\NormalTok{, }
    \DataTypeTok{subtitle=}\StringTok{"(lang, n)"}\NormalTok{, }
    \DataTypeTok{caption=}\StringTok{"Travistorrent"}\NormalTok{) }\OperatorTok{+}
\StringTok{  }\KeywordTok{theme}\NormalTok{(}\DataTypeTok{plot.title =} \KeywordTok{element_text}\NormalTok{(}\DataTypeTok{face=}\StringTok{"bold"}\NormalTok{),}\DataTypeTok{panel.border=}\KeywordTok{element_blank}\NormalTok{())}
\end{Highlighting}
\end{Shaded}

\includegraphics{lab2-checkpoint-1.1_files/figure-latex/unnamed-chunk-2-1.pdf}

\section{Objetivos}\label{objetivos}

Precisamos avaliar um conjunto de variáveis em um exercício de
visualização de correlações de dados através de marcas e canais. O
objetivo de nossa investigação é entender os dados e as relações entre
suas variáveis. Tendo isto em vista, é importante ser capaz de escolher
as marcas e canais adequados para nossa análise.

A partir disto, devemos propor possíveis correlações alvo ligadas às
relações entre variáveis que julguemos interessantes, para em seguida
proceder a análise.

\subsubsection{Regras}\label{regras}

De acordo com o enunciado do Laboratório 2 da disciplina LPCC2, devemos
elaborar pelo menos seis gráficos envolvendo quatro variáveis
diferentes. Cada um com ao menos duas marcas distintas.

\section{Variáveis de estudo}\label{variaveis-de-estudo}

Selecionamos para este relatório quatro variáveis que serão alvo de
comparações, medições e visualizações, são elas: \textbf{sloc\_end},
\textbf{sloc\_med}, \textbf{tests\_per\_kloc} e
\textbf{test\_size\_avg}, outras variáveis também podem ser usadas na
comparação, como \textbf{lang}, \textbf{activity\_period} ou
\textbf{team}. Nosso objetivo é observar seu comportamento, confirmando
ou refutando certas perspectivas de correlação que chamaremos de
questões e são descritas abaixo.

\section{Análise}\label{analise}

A partir daqui analisaremos nossos dados com base em questões levantadas
a partir de correlações entre variáveis. Neste estudo, o principal foco
é a análise visual. Utilizaremos diversas variedades de gráficos
estatísticos, canais de visualização e marcas.

Um canal é uma dimensão usada para converter dados de uma tabela em uma
visualização gráfica. Um canal pode ser de alta ou baixa magnitude,
representando uma eficácia na visualização de informação maior ou menor.

Nosso primeiro objetivo é entender como o volume de código se comporta
ao longo do tempo, e faremos isto comparando as variáveis
\textbf{sloc\_end} e \textbf{sloc\_med} em relação à outras variáveis.
Em seguida tentaremos entender a relação entre \textbf{tests\_per\_kloc}
e \textbf{test\_size\_avg}, sempre fazendo um paralelo com variáveis
significativas como \textbf{lang} e/ou \textbf{team}, priorizando canais
mais eficazes para variáveis alvo de análise.

\subsubsection{\texorpdfstring{\textbf{sloc\_end} e
\textbf{sloc\_med}}{sloc\_end e sloc\_med}}\label{sloc_end-e-sloc_med}

A principal relação entre estas duas variáveis é o tempo.
\textbf{sloc\_med}, como já foi mencionado, mede o volume de código de
um projeto na metade de seu tempo de vida (até o fim da medição dos
dados), já \textbf{sloc\_end} mede o volume total de código no fim do
período de medição. É de se esperar que \textbf{sloc\_med} sejá menor
que \textbf{sloc\_end}, mas qual a proporção desta diferença?

Para começar a responder esta pergunta poderíamos pensar em usar um
gráfico de dispersão (\emph{scatter plot}). O gráfico de dispersão
posiciona todos os elementos observados como pontos em um plano
cartesiano, onde x corresponde a uma variável e y à outra.

Um bom tipo de gráfico para relacionar \textbf{sloc\_med} e
\textbf{sloc\_end} é o \emph{Dumbbell Plot}, que pode ser visto como uma
variação do gráfico de dispersão. Nele o canal ``posicionamento
bidimensional''" está representado na forma de pontos em um plano tal
como o gráfico de dispersão. Porém, o eixo x apresenta duas variáveis e
a progressão de uma até a outra, formando uma linha (o comprimento da
linha é mais um canal). Abaixo está um \emph{Dumbbell Plot} onde x
descreve a variável \textbf{sloc\_med} e a distância entre ela e
\textbf{sloc\_end}. Incluiremos também as variáveis \textbf{lang} e
\textbf{team} nos canais cor dos pontos e dimensão vertical para
posicionar melhor nosso gráfico nos nossos dados.

\begin{Shaded}
\begin{Highlighting}[]
\NormalTok{gg <-}\StringTok{ }\NormalTok{projetos }\OperatorTok
\StringTok{  }\KeywordTok{ggplot}\NormalTok{(}\KeywordTok{aes}\NormalTok{(}\DataTypeTok{x=}\NormalTok{sloc_med, }\DataTypeTok{xend=}\NormalTok{sloc_end, }\DataTypeTok{y=}\NormalTok{team, }\DataTypeTok{group=}\NormalTok{team, }\DataTypeTok{color=}\NormalTok{lang)) }\OperatorTok{+}\StringTok{ }
\StringTok{    }\KeywordTok{geom_dumbbell}\NormalTok{(}\DataTypeTok{size=}\FloatTok{1.0}\NormalTok{, }\DataTypeTok{alpha=}\NormalTok{.}\DecValTok{3}\NormalTok{) }\OperatorTok{+}\StringTok{ }
\StringTok{    }\KeywordTok{scale_y_log10}\NormalTok{()}\OperatorTok{+}\StringTok{ }
\StringTok{    }\KeywordTok{labs}\NormalTok{(}\DataTypeTok{x=}\StringTok{'Código desenvolvida na segunda metade da vida do projeto'}\NormalTok{,  }
      \DataTypeTok{y=}\StringTok{'Tamanho da equipe'}\NormalTok{, }
      \DataTypeTok{title=}\StringTok{"Progressão do Desenvolvimento do Código", }
\StringTok{      subtitle="}\NormalTok{( (sloc_me,sloc_end) ,team)}\StringTok{", }
\StringTok{      caption="}\NormalTok{Travistorrent}\StringTok{") +}
\StringTok{    theme(plot.title = element_text(face="}\NormalTok{bold}\StringTok{"),}
\StringTok{      plot.background=element_rect(fill="}\CommentTok{#ffffff"),}
      \DataTypeTok{panel.background=}\KeywordTok{element_rect}\NormalTok{(}\DataTypeTok{fill=}\StringTok{"#ffffff"}\NormalTok{),}
      \DataTypeTok{legend.background =} \KeywordTok{element_rect}\NormalTok{(}\DataTypeTok{fill=}\StringTok{"#ffffff"}\NormalTok{),}
      \DataTypeTok{panel.grid.minor=}\KeywordTok{element_blank}\NormalTok{(),}
      \DataTypeTok{panel.grid.major.y=}\KeywordTok{element_blank}\NormalTok{(),}
      \DataTypeTok{panel.grid.major.x=}\KeywordTok{element_line}\NormalTok{(}\DataTypeTok{color=}\StringTok{"#a3c4dc"}\NormalTok{),}
      \DataTypeTok{axis.ticks=}\KeywordTok{element_blank}\NormalTok{(),}
      \DataTypeTok{legend.position=}\StringTok{"top"}\NormalTok{,}
      \DataTypeTok{axis.text.x =} \KeywordTok{element_blank}\NormalTok{(),}
      \DataTypeTok{panel.border=}\KeywordTok{element_blank}\NormalTok{())}
\KeywordTok{plot}\NormalTok{(gg)}
\end{Highlighting}
\end{Shaded}

\includegraphics{lab2-checkpoint-1.1_files/figure-latex/unnamed-chunk-3-1.pdf}

Observando os pontos e traços neste gráfico podemos ter a ideia de que
talvez a maior parte do desenvolvimento ocorra entre o início e a
primeira metade do projeto. Isto com base na diferença entre o
comprimento dos traços que descrevem a distância entre
\textbf{sloc\_med} e \textbf{sloc\_end}, e a distância entre
\textbf{sloc\_med} e a origem de x. Porém, é uma conclusão muito
inicial, pois há muitos pontos aglomerados de difícil visualização, além
de alguns pontos com grandes distâncias entre \textbf{sloc\_med} e
\textbf{sloc\_end}.

Uma nova evidência para esta afirmação poderia ser obtida observando não
as duas variáveis separadas, mas uma nova variável composta pela
proporção entre elas. A fórmula ``(sloc\_med*100)/sloc\_end'' nos traz
este valor. Novamente, para posicionar melhor nossa visualização nos
nossos dados, incluiremos as seguintes três variáveis no gráfico;
\textbf{team}, \textbf{activity\_period} e \textbf{lang} nos canais
dimensão horizontal, área do ponto e cor, respectivamente. Um
\emph{boxplot} para cada um de cem \emph{breaks} (quando diferentes de
vazio) ajuda a visualizar a relação.

\begin{Shaded}
\begin{Highlighting}[]
\NormalTok{projetos }\OperatorTok
\StringTok{  }\KeywordTok{ggplot}\NormalTok{(}\KeywordTok{aes}\NormalTok{(}\DataTypeTok{x=}\KeywordTok{cut}\NormalTok{(team, }\DataTypeTok{breaks=}\DecValTok{100}\NormalTok{), }\DataTypeTok{y=}\NormalTok{(sloc_med}\OperatorTok{*}\DecValTok{100}\NormalTok{)}\OperatorTok{/}\NormalTok{sloc_end)) }\OperatorTok{+}
\StringTok{  }\KeywordTok{geom_count}\NormalTok{(}\KeywordTok{aes}\NormalTok{(}\DataTypeTok{color=}\NormalTok{lang,}\DataTypeTok{size=}\NormalTok{activity_period),}\DataTypeTok{alpha =}\NormalTok{ .}\DecValTok{2}\NormalTok{)}\OperatorTok{+}\StringTok{ }
\StringTok{  }\KeywordTok{geom_boxplot}\NormalTok{(}\DataTypeTok{outlier.alpha =}\NormalTok{ .}\DecValTok{0}\NormalTok{, }\DataTypeTok{fill=}\StringTok{'#000000'}\NormalTok{, }\DataTypeTok{color =} \StringTok{"darkcyan"}\NormalTok{) }\OperatorTok{+}
\StringTok{    }\KeywordTok{labs}\NormalTok{(}\DataTypeTok{x=}\StringTok{'Tamanho da equipe'}\NormalTok{, }
      \DataTypeTok{y=}\StringTok{'Código desenvolvida até metade da vida do projeto (%)'}\NormalTok{, }
      \DataTypeTok{color=}\StringTok{'Linguagem'}\NormalTok{, }
      \DataTypeTok{size=}\StringTok{'Período de atividade'}\NormalTok{, }
      \DataTypeTok{title=}\StringTok{"Progressão do Desenvolvimento do Código", }
\StringTok{      subtitle="}\NormalTok{(team, (sloc_med}\OperatorTok{*}\DecValTok{100}\NormalTok{) }\OperatorTok{/}\StringTok{ }\NormalTok{sloc_end)}\StringTok{", }
\StringTok{      caption="}\NormalTok{Travistorrent}\StringTok{") +}
\StringTok{  theme(plot.title = element_text( face="}\NormalTok{bold}\StringTok{"),}
\StringTok{    plot.background=element_rect(fill="}\CommentTok{#ffffff"),}
    \DataTypeTok{legend.background =} \KeywordTok{element_rect}\NormalTok{(}\DataTypeTok{fill=}\StringTok{"#ffffff"}\NormalTok{),}
    \DataTypeTok{panel.background=}\KeywordTok{element_rect}\NormalTok{(}\DataTypeTok{fill=}\StringTok{"#ffffff"}\NormalTok{),}
    \DataTypeTok{panel.grid.minor=}\KeywordTok{element_blank}\NormalTok{(),}
    \DataTypeTok{panel.grid.major.y=}\KeywordTok{element_blank}\NormalTok{(),}
    \DataTypeTok{panel.grid.major.x=}\KeywordTok{element_line}\NormalTok{(),}
    \DataTypeTok{axis.ticks=}\KeywordTok{element_blank}\NormalTok{(),}
    \DataTypeTok{axis.text.x =} \KeywordTok{element_blank}\NormalTok{(),}
    \DataTypeTok{legend.position=}\StringTok{"top"}\NormalTok{,}
    \DataTypeTok{panel.border=}\KeywordTok{element_blank}\NormalTok{())}
\end{Highlighting}
\end{Shaded}

\includegraphics{lab2-checkpoint-1.1_files/figure-latex/unnamed-chunk-4-1.pdf}

No gráfico acima é possível perceber que a maior parte dos pontos está
acima de 60\%, o que nos leva a concluir que geralmente mais da metade
do código é escrito no início do projeto. Além disto, é frequênte, de
acordo com o gráfico, projetos com 80\% ou 90\% do desenvolvimento
ocorrido na primeira metade de sua vida. Observando os \emph{bosplots}
pode-se perceber apenas um \emph{box} abaixo de 50\%, e com apenas um
projeto.

Uma visualização mais clara talvez possa vir com um gráfico de dispersão
(\emph{scatter plot}) simples, que descreve a posição de cada projeto em
um plano cartesiano onde x é o \textbf{sloc\_end} e y é
\textbf{sloc\_med}. Novamente, incluiremos também as variáveis
\textbf{lang} e \textbf{team} nos canais cor e área dos pontos para
posicionar melhor nosso gráfico nos nossos dados.

\begin{Shaded}
\begin{Highlighting}[]
\NormalTok{projetos }\OperatorTok\StringTok{ }
\StringTok{    }\KeywordTok{ggplot}\NormalTok{(}\KeywordTok{aes}\NormalTok{(}\DataTypeTok{x=}\NormalTok{ sloc_end, }\DataTypeTok{y=}\NormalTok{ sloc_med , }\DataTypeTok{color =}\NormalTok{ lang, }\DataTypeTok{size=}\NormalTok{team),}\DataTypeTok{na.rm=}\OtherTok{TRUE}\NormalTok{) }\OperatorTok{+}\StringTok{ }
\StringTok{    }\KeywordTok{scale_x_log10}\NormalTok{()}\OperatorTok{+}\StringTok{ }
\StringTok{    }\KeywordTok{scale_y_log10}\NormalTok{()}\OperatorTok{+}\StringTok{ }
\StringTok{    }\KeywordTok{geom_point}\NormalTok{(}\DataTypeTok{alpha =}\NormalTok{ .}\DecValTok{3}\NormalTok{) }\OperatorTok{+}\StringTok{ }
\StringTok{        }\KeywordTok{labs}\NormalTok{(}\DataTypeTok{x=}\StringTok{'Total de linhas de código (final)'}\NormalTok{,  }
             \DataTypeTok{y=}\StringTok{'Total de linhas de código (mediana)'}\NormalTok{, }
             \DataTypeTok{title=}\StringTok{'Progressão do volume do código', }
\StringTok{             color='}\NormalTok{Linguagem}\StringTok{', }
\StringTok{             size='}\NormalTok{Tamanho da equipe}\StringTok{', }
\StringTok{             subtitle='}\NormalTok{(sloc_end, sloc_med)}\StringTok{', }
\StringTok{             caption="Travistorrent") +}
\StringTok{  theme_bw() +}
\StringTok{  theme(plot.title = element_text(face="bold"),}
\StringTok{        panel.border=element_blank(),}
\StringTok{        legend.position="top",}
\StringTok{        axis.text.x = element_blank(),}
\StringTok{        axis.text.y = element_blank(),}
\StringTok{        plot.margin=unit(c(10,0,0,0),"points"))}
\end{Highlighting}
\end{Shaded}

\includegraphics{lab2-checkpoint-1.1_files/figure-latex/unnamed-chunk-5-1.pdf}

Veja a eficácia do canal ``posicionamento no plano'', apenas com base no
gráfico acima já podemos perceber claramente dois atributos desta
relação. O primeiro aponta que de fato, como já poderíamos supor,
\textbf{sloc\_end} é maior que \textbf{sloc\_med} aparentemente em todos
os casos. Perceba que não há nenhum ponto acima da diagonal principal do
nosso gráfico. Neste momento, não podemos garantir que esta afirmação é
verdade para todos os casos, pois os canais de visualização geralmente
proveem apenas uma direção para a análise. Em buscar de uma resposta
categórica, precisamos de um modelo. Veja abaixo.

\begin{Shaded}
\begin{Highlighting}[]
\NormalTok{projetos }\OperatorTok\StringTok{ }
\StringTok{  }\KeywordTok{filter}\NormalTok{(sloc_med }\OperatorTok{>}\StringTok{ }\NormalTok{sloc_end) }\OperatorTok
\StringTok{  }\KeywordTok{summarise}\NormalTok{(}\StringTok{`}\DataTypeTok{Total de projetos com mais código na metade de seu tempo de vida que o final}\StringTok{`}\NormalTok{=}\KeywordTok{n}\NormalTok{())}
\end{Highlighting}
\end{Shaded}

\begin{verbatim}
## # A tibble: 1 x 1
##   `Total de projetos com mais código na metade de seu tempo de vida que o~
##                                                                      <int>
## 1                                                                        0
\end{verbatim}

Agora podemos afirmar categoricamente que nenhum projeto em nossa base
encolheu (refatoramento) entre a metade e o fim de sua vida. O segundo
atributo da relação entre estas duas variáveis diz respeito à
linearidade. É possível perceber que a figura composta pelos pontos no
gráfico forma uma linha muito nítida à quase que exatos 45° de
inclinação, bem em cima da diagonal principal do gráfico. Isto sugere
que os valores de \textbf{sloc\_med} e \textbf{sloc\_end} são muito
parecidos, o que nos leva a crer que a maior parte do desenvolvimento
ocorra na primeira metade da vida dos projetos na nossa base de dados,
como os dois gráficos anteriores já demonstravam.

Novamente os canais de visualização nos apontam a direção da análise,
mas para responder se a maior parte do desenvolvimento realmente ocorre
no início do projeto devemos recorrer aos modelos. Através de um
coeficiente de correlação linear entre \textbf{sloc\_med} e
\textbf{sloc\_end} podemos ter a confirmação que buscamos. Nesta
correlação sempre retornará um valor entre -1 e 1, onde -1 representa
uma perfeita correlação decrescente, 0 a ausência de correlação e 1
representa uma correlação crescente forte. Aqui calculamos a correção
entre estas variáveis de três métodos diferentes.

\begin{Shaded}
\begin{Highlighting}[]
\NormalTok{projetos }\OperatorTok
\StringTok{    }\KeywordTok{summarise}\NormalTok{(}\DataTypeTok{pearson =} \KeywordTok{cor}\NormalTok{(sloc_end, sloc_med, }\DataTypeTok{method =} \StringTok{"pearson"}\NormalTok{),}
              \DataTypeTok{spearman =} \KeywordTok{cor}\NormalTok{(sloc_end, sloc_med, }\DataTypeTok{method =} \StringTok{"spearman"}\NormalTok{),}
              \DataTypeTok{kendall =} \KeywordTok{cor}\NormalTok{(sloc_end, sloc_med, }\DataTypeTok{method =} \StringTok{"kendall"}\NormalTok{))}
\end{Highlighting}
\end{Shaded}

\begin{verbatim}
## # A tibble: 1 x 3
##   pearson spearman kendall
##     <dbl>    <dbl>   <dbl>
## 1   0.980    0.981   0.906
\end{verbatim}

A tabela apresenta os três valores referentes as três formas de calcular
a correlação entre as variáveis. Perceba que todos apresentam uma
correlação superior a 0.9, o que indica uma forte correlação. Esta
evidência responde nossa atual dúvida. De fato, como os gráficos
mostraram, a maior parte do desenvolvimento ocorre na primeira metade do
projeto.

\paragraph{Conclusão}\label{conclusao}

Podemos concluir com base nos gráficos e resultados matemáticos, que
nenhum projeto teve seu código refatorado a ponto de encolher da metade
do tempo de vida do projeto até o fim e que a maior parte do
desenvolvimento ocorre, em média, na primeira metade do tempo de vida do
projeto.

Como ameaça à validade desta conclusão, podemos mencionar que a variável
\textbf{sloc\_end} não representa, de fato, o fim da vida do projeto,
mas o fim da medição. Alguns projetos podem ter entrado em produção
(quando passar a sofrer muito menos alterações) antes mesmo do momento
em que \textbf{sloc\_med} foi registrado, outros não terem atingido a
maturidade mesmo ao final do tempo total.

\subsubsection{\texorpdfstring{\textbf{tests\_per\_kloc} e
\textbf{test\_size\_avg}}{tests\_per\_kloc e test\_size\_avg}}\label{tests_per_kloc-e-test_size_avg}

Uma relação presente entre estas duas variáveis é que ambas aumentam o
volume total do código destinado a testes. \textbf{tests\_per\_kloc}
mede o volume de testes por mil linhas de código, já
\textbf{test\_size\_avg} mede o tamanho médio de cada caso de teste em
um projeto. Casos de teste mais complexos demandam mais código
(geralmente), mas a maioria dos casos são descritos em poucas linhas.
Por isto é de se esperar que o tamanho médio destes casos não seja muito
grande. Porém, como veremos adiante, não é exatamente o que acontece.
Muitos projetos têm, em média, casos de teste demasiadamente grandes. Um
caso de teste grande pode representar um problema complexo ou mais de um
caso de teste em um só. Com isto temos a pergunta: Casos de teste muito
grandes levam a menos casos de testes?

O posicionamento no espaço bidimensional é o canal de mais alta
magnitude em gráficos estatísticos. Por isso o gráfico de dispersão,
quando aplicável, será sempre uma boa opção para iniciar uma análise.
Este possibilita não apenas a visualização dos valores numéricos em
escala, mas também a marcação da identidade de outras variáveis
categóricas.

Em busca da resposta para a pergunta proposta partiremos de um gráfico
de dispersão comparativo, onde x é \textbf{tests\_per\_kloc} e y é
\textbf{test\_size\_avg}. Mais uma vez, incluiremos também as variáveis
\textbf{lang} e \textbf{team} nos canais cor e área dos pontos para
posicionar melhor nosso gráfico nos nossos dados.

\begin{Shaded}
\begin{Highlighting}[]
\NormalTok{projetos }\OperatorTok\StringTok{ }
\StringTok{  }\KeywordTok{ggplot}\NormalTok{(}\KeywordTok{aes}\NormalTok{(}\DataTypeTok{x=}\NormalTok{ tests_per_kloc, }\DataTypeTok{y=}\NormalTok{ test_size_avg)) }\OperatorTok{+}\StringTok{ }
\StringTok{  }\KeywordTok{scale_x_log10}\NormalTok{()}\OperatorTok{+}\StringTok{ }
\StringTok{  }\KeywordTok{scale_y_log10}\NormalTok{()}\OperatorTok{+}\StringTok{ }
\StringTok{  }\KeywordTok{geom_point}\NormalTok{(}\KeywordTok{aes}\NormalTok{(}\DataTypeTok{x=}\NormalTok{ tests_per_kloc, }\DataTypeTok{y=}\NormalTok{ test_size_avg, }\DataTypeTok{size=}\NormalTok{team, }\DataTypeTok{color=}\NormalTok{lang), }\DataTypeTok{alpha =}\NormalTok{ .}\DecValTok{3}\NormalTok{) }\OperatorTok{+}
\StringTok{  }\KeywordTok{stat_smooth}\NormalTok{(}\DataTypeTok{method=}\StringTok{'loess'}\NormalTok{, }\DataTypeTok{se=}\NormalTok{F)}\OperatorTok{+}
\StringTok{  }\KeywordTok{labs}\NormalTok{(}\DataTypeTok{x=}\StringTok{'Testes por 1000 linhas de código',  }
\StringTok{    y='}\NormalTok{Tamanho dos casos de teste}\StringTok{', }
\StringTok{    caption="Travistorrent",}
\StringTok{    size="Tamanho da equipe",}
\StringTok{    color="Linguagem",}
\StringTok{    title="Volume de teste x Tamanho dos testes",}
\StringTok{    subtitle="(tests_per_kloc, test_size_avg)") +}
\StringTok{  theme(panel.border=element_blank(),}
\StringTok{    legend.position="bottom",}
\StringTok{    plot.margin=unit(c(10,0,0,0),"points"))}
\end{Highlighting}
\end{Shaded}

\includegraphics{lab2-checkpoint-1.1_files/figure-latex/unnamed-chunk-8-1.pdf}

É possível observar uma certa correlação linear entre as duas variáveis,
o que sugere uma conexão (ainda que pequena). A curva que descreve a
distribuição dos pontos torna isto um pouco mais nítido. Novamente, por
mais forte que canais de posicionamento sejam, somente o cálculo desta
correlação pode responder à esta pergunta definitivamente. Por enquanto
nosso interesse é apenas na observação dos dados.

Os canais de posicionamento não se encontram restritos apenas ao uso
convencional nos eixos x e y. Como mostramos no \emph{Dumbbell Plot}, é
possível ir além. Partiremos para explorar a visualização de gráficos
justapostos, posicionados em escala de modo a possibilitar a comparação
da distribuição dos valores nos dados. Veja o mesmo gráfico exibido
acima, mas agora justaposto a gráficos de densidade acumulada alinhados
de modo a explorarmos o fator posicionamento.

\begin{Shaded}
\begin{Highlighting}[]
\NormalTok{p1 <-}\StringTok{ }\NormalTok{projetos }\OperatorTok\StringTok{ }
\StringTok{  }\KeywordTok{filter}\NormalTok{(test_size_avg }\OperatorTok{>}\StringTok{ }\FloatTok{0.0}\NormalTok{, tests_per_kloc }\OperatorTok{>}\StringTok{ }\FloatTok{0.0}\NormalTok{) }\OperatorTok\StringTok{ }
\StringTok{  }\KeywordTok{ggplot}\NormalTok{(}\KeywordTok{aes}\NormalTok{(}\DataTypeTok{x=}\NormalTok{ tests_per_kloc, }\DataTypeTok{y=}\NormalTok{ test_size_avg, }\DataTypeTok{size=}\NormalTok{team, }\DataTypeTok{color=}\NormalTok{lang)) }\OperatorTok{+}\StringTok{ }
\StringTok{  }\KeywordTok{geom_point}\NormalTok{(}\DataTypeTok{alpha =}\NormalTok{ .}\DecValTok{3}\NormalTok{) }\OperatorTok{+}
\StringTok{  }\KeywordTok{scale_x_log10}\NormalTok{()}\OperatorTok{+}\StringTok{ }
\StringTok{  }\KeywordTok{scale_y_log10}\NormalTok{()}\OperatorTok{+}\StringTok{ }
\StringTok{  }\KeywordTok{labs}\NormalTok{(}\DataTypeTok{x=}\StringTok{'Testes por 1000 linhas de código',  }
\StringTok{    y='}\NormalTok{Tamanho dos casos de teste}\StringTok{', }
\StringTok{    title=NULL, }
\StringTok{    subtitle=NULL, }
\StringTok{    size="Tamanho da equipe",}
\StringTok{    color="Linguagem",}
\StringTok{    caption="Travistorrent") +}
\StringTok{  theme_bw() +}
\StringTok{  theme(panel.border=element_blank(),}
\StringTok{    legend.position="bottom",}
\StringTok{    plot.margin=unit(c(10,0,0,0),"points"))}



\StringTok{p2 <- projetos %>% ggplot(aes(x=tests_per_kloc,colour=lang,fill=lang)) + }
\StringTok{  geom_density(alpha=0.5) + }
\StringTok{  scale_x_log10()+}
\StringTok{  labs(x=NULL,}
\StringTok{    y=NULL,}
\StringTok{    title="Volume de teste x Tamanho dos testes",}
\StringTok{    subtitle="(tests_per_kloc, test_size_avg)") +}
\StringTok{  theme_bw() +}
\StringTok{  theme(plot.background=element_rect(fill="#ffffff"),}
\StringTok{    legend.background = element_rect(fill="#ffffff"),}
\StringTok{    panel.background=element_rect(fill="#ffffff"),}
\StringTok{    panel.grid.minor=element_blank(),}
\StringTok{    panel.grid.major.y=element_blank(),}
\StringTok{    panel.grid.major.x=element_blank(),}
\StringTok{    axis.ticks=element_blank(),}
\StringTok{    axis.text.x = element_blank(),}
\StringTok{    axis.text.y = element_blank(),}
\StringTok{    panel.border=element_blank(),}
\StringTok{    legend.position="none",}
\StringTok{    plot.margin=unit(c(20,0,0,20),"points"))}

\StringTok{p3 <- projetos %>% ggplot(aes(x=test_size_avg,colour=lang,fill=lang)) + }
\StringTok{  geom_density(alpha=0.5) + }
\StringTok{  scale_x_log10()+}
\StringTok{  coord_flip()  + }
\StringTok{  labs(x=NULL,  }
\StringTok{    y=NULL) +}
\StringTok{  theme(plot.background=element_rect(fill="#ffffff"),}
\StringTok{    legend.background = element_rect(fill="#ffffff"),}
\StringTok{    panel.background=element_rect(fill="#ffffff"),}
\StringTok{    panel.grid.minor=element_blank(),}
\StringTok{    panel.grid.major.y=element_blank(),}
\StringTok{    panel.grid.major.x=element_blank(),}
\StringTok{    axis.ticks=element_blank(),}
\StringTok{    axis.text.x = element_blank(),}
\StringTok{    axis.text.y = element_blank(),}
\StringTok{    panel.border=element_blank(),}
\StringTok{    legend.position="none",}
\StringTok{    plot.margin=unit(c(-10,60,70,0),"points"))}

\StringTok{grid.arrange(arrangeGrob(p2,ncol=2,widths=c(3,1)),}
\StringTok{    arrangeGrob(p1,p3,ncol=2,widths=c(3,1)),}
\StringTok{    heights=c(1,3))}
\end{Highlighting}
\end{Shaded}

\includegraphics{lab2-checkpoint-1.1_files/figure-latex/unnamed-chunk-9-1.pdf}

Observe que ainda não adicionamos nenhuma nova variável ao gráfico,
apenas a densidade acumulada de cada eixo. Isto nos deu mais evidências
de que ambas se acumulam em valores médios, o sugere um baixo desvio
padrão. Uma forma de percebermos isto com mais segurança por meio de
visualizações é utilizando \emph{bloxplots} ao invés de distribuição
acumulada.

\begin{Shaded}
\begin{Highlighting}[]
\NormalTok{p7 <-}\StringTok{ }\NormalTok{projetos }\OperatorTok\StringTok{ }
\StringTok{  }\KeywordTok{filter}\NormalTok{(test_size_avg }\OperatorTok{>}\StringTok{ }\FloatTok{0.0}\NormalTok{, tests_per_kloc }\OperatorTok{>}\StringTok{ }\FloatTok{0.0}\NormalTok{) }\OperatorTok\StringTok{ }
\StringTok{  }\KeywordTok{ggplot}\NormalTok{(}\KeywordTok{aes}\NormalTok{(}\DataTypeTok{x=}\NormalTok{ tests_per_kloc, }\DataTypeTok{y=}\NormalTok{ test_size_avg, }\DataTypeTok{size=}\NormalTok{team, }\DataTypeTok{color=}\NormalTok{lang)) }\OperatorTok{+}\StringTok{ }
\StringTok{    }\KeywordTok{geom_point}\NormalTok{(}\DataTypeTok{alpha =}\NormalTok{ .}\DecValTok{3}\NormalTok{) }\OperatorTok{+}
\StringTok{    }\KeywordTok{scale_x_log10}\NormalTok{()}\OperatorTok{+}\StringTok{ }
\StringTok{    }\KeywordTok{scale_y_log10}\NormalTok{()}\OperatorTok{+}\StringTok{ }
\StringTok{    }\KeywordTok{labs}\NormalTok{(}\DataTypeTok{x=}\StringTok{'Testes por 1000 linhas de código',  }
\StringTok{      y='}\NormalTok{Tamanho dos casos de teste}\StringTok{', }
\StringTok{      title=NULL, }
\StringTok{      subtitle=NULL, }
\StringTok{      size="Tamanho da equipe",}
\StringTok{      color="Linguagem",}
\StringTok{      caption="Travistorrent") +}
\StringTok{    theme_bw() +}
\StringTok{    theme(panel.border=element_blank(),}
\StringTok{      legend.position="bottom",}
\StringTok{      plot.margin=unit(c(10,0,0,0),"points"))}


\StringTok{p8 <- projetos %>%}
\StringTok{  ggplot(aes(y = tests_per_kloc,x='}\NormalTok{hjhj}\StringTok{',fill=lang)) +}
\StringTok{    scale_y_log10()+}
\StringTok{    coord_flip()  + }
\StringTok{    geom_boxplot(outlier.alpha = .0, color = "#808080") +}
\StringTok{    labs(x=NULL,  }
\StringTok{      y=NULL,}
\StringTok{      title="Volume de teste x Tamanho dos testes",}
\StringTok{      subtitle="(tests_per_kloc, test_size_avg)") +}
\StringTok{    theme(plot.background=element_rect(fill="#ffffff"),}
\StringTok{      legend.background = element_rect(fill="#ffffff"),}
\StringTok{      panel.background=element_rect(fill="#ffffff"),}
\StringTok{      panel.grid.minor=element_blank(),}
\StringTok{      panel.grid.major.y=element_blank(),}
\StringTok{      panel.grid.major.x=element_blank(),}
\StringTok{      axis.ticks=element_blank(),}
\StringTok{      axis.text.y = element_blank(),}
\StringTok{      axis.text.x = element_blank(),}
\StringTok{      panel.border=element_blank(),}
\StringTok{      legend.position="none",}
\StringTok{      plot.margin=unit(c(0,-20,0,20),"points"))}

\StringTok{p9 <- projetos %>%}
\StringTok{  ggplot(aes(x = '}\NormalTok{um label}\StringTok{',y = test_size_avg,fill=lang)) +}
\StringTok{    scale_y_log10()+}
\StringTok{    geom_boxplot(outlier.alpha = .0, color = "#808080") +}
\StringTok{    labs(x=NULL,  }
\StringTok{      y=NULL) +}
\StringTok{    theme(plot.background=element_rect(fill="#ffffff"),}
\StringTok{      legend.background = element_rect(fill="#ffffff"),}
\StringTok{      panel.background=element_rect(fill="#ffffff"),}
\StringTok{      panel.grid.minor=element_blank(),}
\StringTok{      panel.grid.major.y=element_blank(),}
\StringTok{      panel.grid.major.x=element_blank(),}
\StringTok{      axis.ticks=element_blank(),}
\StringTok{      axis.text.x = element_blank(),}
\StringTok{      axis.text.y = element_blank(),}
\StringTok{      panel.border=element_blank(),}
\StringTok{      legend.position="none",}
\StringTok{      plot.margin=unit(c(0,60,70,0),"points"))}

\StringTok{grid.arrange(arrangeGrob(p8,ncol=2,widths=c(3,1)),}
\StringTok{   arrangeGrob(p7,p9,ncol=2,widths=c(3,1)),}
\StringTok{   heights=c(1,3))}
\end{Highlighting}
\end{Shaded}

\includegraphics{lab2-checkpoint-1.1_files/figure-latex/unnamed-chunk-10-1.pdf}

Perceba que os \emph{boxs} são curtos em relação a amplitude de valores
das variáveis, o que aponta para um acumulo de valores em torno da
mediana e, por sua vez, um provável desvio padrão baixo para ambas as
variáveis. Mas ainda não sabemos se testes grandes levam a poucos
testes.

Podemos tentar justapor o gráfico de dispersão destas duas variáveis em
relação a outras. Usaremos agora o canal de posicionamento espacial para
comparar estas variáveis com a variável \textbf{team} em um canal mais
eficazes que a o tamanho das marcas (na área das marcas exibiremos
\textbf{activity\_period}, apenas para ampliar a visualização dos dados
no nosso gráfico). Observe abaixo.

\begin{Shaded}
\begin{Highlighting}[]
\NormalTok{p7 <-}\StringTok{ }\NormalTok{projetos }\OperatorTok\StringTok{ }
\StringTok{  }\KeywordTok{filter}\NormalTok{(test_size_avg }\OperatorTok{>}\StringTok{ }\FloatTok{0.0}\NormalTok{, tests_per_kloc }\OperatorTok{>}\StringTok{ }\FloatTok{0.0}\NormalTok{) }\OperatorTok\StringTok{ }
\StringTok{  }\KeywordTok{ggplot}\NormalTok{(}\KeywordTok{aes}\NormalTok{(}\DataTypeTok{x=}\NormalTok{ tests_per_kloc, }\DataTypeTok{y=}\NormalTok{ test_size_avg, }\DataTypeTok{size=}\NormalTok{activity_period, }\DataTypeTok{color=}\NormalTok{lang)) }\OperatorTok{+}\StringTok{ }
\StringTok{    }\KeywordTok{geom_point}\NormalTok{(}\DataTypeTok{alpha =}\NormalTok{ .}\DecValTok{3}\NormalTok{) }\OperatorTok{+}
\StringTok{    }\KeywordTok{scale_x_log10}\NormalTok{()}\OperatorTok{+}\StringTok{ }
\StringTok{    }\KeywordTok{scale_y_log10}\NormalTok{()}\OperatorTok{+}\StringTok{ }
\StringTok{    }\KeywordTok{labs}\NormalTok{(}\DataTypeTok{x=}\StringTok{'Testes por 1000 linhas de código',  }
\StringTok{      y='}\NormalTok{Tamanho dos casos de teste}\StringTok{', }
\StringTok{      title=NULL, }
\StringTok{      subtitle=NULL, }
\StringTok{      size="Idade do projeto",}
\StringTok{      color="Linguagem",}
\StringTok{      caption="Travistorrent") +}
\StringTok{    theme_bw() +}
\StringTok{    theme(panel.border=element_blank(),}
\StringTok{      legend.position="bottom",}
\StringTok{      plot.margin=unit(c(10,0,0,0),"points"))}


\StringTok{p8 <- projetos %>%}
\StringTok{  filter(test_size_avg > 0.0, tests_per_kloc > 0.0) %>% }
\StringTok{  ggplot(aes(x=tests_per_kloc,y = team   )) +}
\StringTok{    scale_x_log10()+ }
\StringTok{    scale_y_log10()+ }
\StringTok{    geom_point(alpha = .3, size=.0) +}
\StringTok{    labs(x=NULL,  }
\StringTok{      y=NULL,}
\StringTok{      title="Volume de teste x Tamanho dos testes",}
\StringTok{      subtitle="(tests_per_kloc, test_size_avg)") +}
\StringTok{    theme(plot.background=element_rect(fill="#ffffff"),}
\StringTok{      legend.background = element_rect(fill="#ffffff"),}
\StringTok{      panel.background=element_rect(fill="#ffffff"),}
\StringTok{      panel.grid.minor=element_blank(),}
\StringTok{      panel.grid.major.y=element_blank(),}
\StringTok{      panel.grid.major.x=element_blank(),}
\StringTok{      axis.ticks=element_blank(),}
\StringTok{      axis.text.y = element_blank(),}
\StringTok{      axis.text.x = element_blank(),}
\StringTok{      panel.border=element_blank(),}
\StringTok{      legend.position="none",}
\StringTok{      plot.margin=unit(c(0,0,0,30),"points"))}

\StringTok{p9 <- projetos %>%}
\StringTok{  filter(test_size_avg > 0.0) %>%}
\StringTok{  ggplot(aes(x=test_size_avg, y = team)) +}
\StringTok{    scale_x_log10()+ }
\StringTok{    scale_y_log10()+ }
\StringTok{    geom_point(alpha = .3, size=.0) +}
\StringTok{    coord_flip()  + }
\StringTok{    labs(x=NULL,  }
\StringTok{      y=NULL) +}
\StringTok{    theme(plot.background=element_rect(fill="#ffffff"),}
\StringTok{      legend.background = element_rect(fill="#ffffff"),}
\StringTok{      panel.background=element_rect(fill="#ffffff"),}
\StringTok{      panel.grid.minor=element_blank(),}
\StringTok{      panel.grid.major.y=element_blank(),}
\StringTok{      panel.grid.major.x=element_blank(),}
\StringTok{      axis.text.x = element_blank(),}
\StringTok{      axis.text.y = element_blank(),}
\StringTok{      axis.ticks=element_blank(),}
\StringTok{      panel.border=element_blank(),}
\StringTok{      legend.position="none",}
\StringTok{      plot.margin=unit(c(10,-20,85,0),"points"))}

\StringTok{grid.arrange(arrangeGrob(p8,ncol=2,widths=c(3,1)),}
\StringTok{    arrangeGrob(p7,p9,ncol=2,widths=c(3,1)),}
\StringTok{    heights=c(1,3))}
\end{Highlighting}
\end{Shaded}

\includegraphics{lab2-checkpoint-1.1_files/figure-latex/unnamed-chunk-11-1.pdf}

Veja que, mesmo através de canais eficazes, não conseguimos ver
correlação entre estas variáveis \textbf{test\_size\_avg} e
\textbf{tests\_per\_kloc}, e a variável \textbf{team}. Nesta caso o
excesso de informação chega a comprometer a visualização (5 variáveis
diferentes no mesmo gráfico).

Para responder nosso questionamento nos restou calcular o coeficiente de
correlação linear para as duas variáveis do nosso estudo. Novamente
usaremos os mesmos três métodos distintos (Pearson, Spearman e Kendall).

\begin{Shaded}
\begin{Highlighting}[]
\NormalTok{projetos }\OperatorTok
\StringTok{  }\KeywordTok{summarise}\NormalTok{(}\DataTypeTok{pearson =} \KeywordTok{cor}\NormalTok{(tests_per_kloc, test_size_avg, }\DataTypeTok{method =} \StringTok{"pearson"}\NormalTok{),}
    \DataTypeTok{spearman =} \KeywordTok{cor}\NormalTok{(tests_per_kloc, test_size_avg, }\DataTypeTok{method =} \StringTok{"spearman"}\NormalTok{),}
    \DataTypeTok{kendall =} \KeywordTok{cor}\NormalTok{(tests_per_kloc, test_size_avg, }\DataTypeTok{method =} \StringTok{"kendall"}\NormalTok{))}
\end{Highlighting}
\end{Shaded}

\begin{verbatim}
## # A tibble: 1 x 3
##   pearson spearman kendall
##     <dbl>    <dbl>   <dbl>
## 1 -0.0770   -0.475  -0.378
\end{verbatim}

\paragraph{Conclusão}\label{conclusao-1}

Com base nos gráficos, sobretudo a dispersão dos pontos no plano, e no
resultado dos coeficientes, podemos concluir que estas duas variáveis
possuem uma certa correlação linear inversa, porém fraca. O coeficiente
de Pearson aponta para a ausência de correlação, porém Spearman e
Kendall apresentam uma correlação muito fraca.

\section{Conclusão final}\label{conclusao-final}

Pudemos perceber a eficácia de diversos canais de visualização e extrair
informação dos nossos dados com base neles. Verificamos que nossas
observações foram mais frutíferas quando utilizamos gráficos de
dispersão, o que argumenta a favor da grande importância deste
instrumento.

A fraca, mas nítida correlação linear entre \textbf{tests\_per\_kloc} e
\textbf{test\_size\_avg} visível nos gráficos, mas não detectada pelos
coeficientes de dispersão nos mostra que modelos matemáticos não são o
bastante em certos casos. Já a indiscutível correlação entre
\textbf{sloc\_end} e \textbf{sloc\_med} aponta que quando os gráficos
cumprem o seu papel de apontar o caminho, é o modelo matemático que
confirmar o resultado.


\end{document}
